\documentclass[a4paper,12pt]{article}

\usepackage[utf8]{inputenc}
\usepackage[polish]{babel}
\usepackage[OT4]{polski}

\usepackage[colorlinks=true,linkcolor=blue,urlcolor=blue]{hyperref}

\usepackage[pdftex]{graphicx}

%\linespread{1.2}


\title{Projekt z Metod Odkrywania Wiedzy\\Dokumentacja wstępna\\\Large{Proste algorytmy klasyfikacji tekstu. Porównania ze standardowymi algorytmami klasyfikacji dostępnymi~w~R.}\\}

\author{Marcin Chwedczuk, Piotr Monarski}
\date{}

\begin{document}
\maketitle

\section{Wstęp}

W ramach projektu z przedmiotu Metody Odkrywania Wiedzy 
postanowiliśmy zająć się klasyfikacją tekstów. 
Zadanie jakie przed sobą postawiliśmy polega na przypisaniu
dziedziny nauki z jaką jest związany dany artykuł naukowy
na podstawie jego abstraktu (nazywanego w dalszej części dokumentu
streszczeniem).

W trakcie projektu zostaną zaimplementowane
algorytm kNN oraz naiwny klasyfikator bayesowski,
zostaną także stworzone procedury wstępnie przetwarzające
tekst (związane z algorytmem TF-IDF).

W ramach części analitycznej postaramy się odpowiedzieć na
następujące pytania:
\begin{enumerate}
	\item
		Który z testowanych algorytmów 
		najlepiej radzi sobie z postawionym
		problemem klasyfikacji
	\item
		Jakie czynniki mają decydujący wpływ na jakość
		uzyskiwanych hipotez
\end{enumerate}

\section{Dane uczące oraz testowe}

	Jako dane uczące oraz testowe postanowiliśmy wykorzystać
	zbiór \textit{NSF Research Award Abstracts 1990-2003 Data Set}
	wchodzący w skład repozytorium UCI.
	Zbiór ten zawiera 129 tysięcy streszczeń artykułów naukowych 
	w języku angielskim,
	które zostały nagrodzone przez National Science Foundation.
	Każdemu streszczeniu towarzyszy szereg informacji zawierających
	między innymi nazwiska autorów artykułu, tytuł artykułu oraz
	dziedzinę zastosowań (przy czym artykuł może mieć więcej 
	niż jedną dziedzinę zastosowań).
	Łączny rozmiar nieskompresowanego zbioru danych to ponad 400 MB.
	
	Ponieważ oryginalne dane zawierają zbyt wiele 
	niepotrzebnych informacji
	więc zostaną wstępnie przetworzone do zbioru 
	par $(F, T)$. Gdzie $F$ oznaczać będzie listę dziedzin artykułu
	a $T$ tekst streszczenia artykułu.
	
	Po przetworzeniu danych może okazać się konieczne odrzucenie
	części z nich - tj. usunięcie tekstów należących do tych kategorii 
	które są reprezentowane w niedostatecznych sposób (np. pokrywają
	mniej niż 500 streszczeń). Podobnie może okazać się konieczne 
	scalenie ze sobą niektórych kategorii np. klasy ,,Computer Science \& Engineering'' oraz ,,Engineering \& Computer Science'' występują w danych jako dwa niezależne byty.
	
	Zbiór testowy będzie stanowił co najmniej 20\% 
	początkowych danych i zostanie
	wybrany w sposób losowy. 
	Pozostałe dane będą stanowić zbiór uczący.
	
	
\section{Wstępne przetworzenie tekstu}
	
	Tekst streszczeń artykułów zostanie wstępnie przetworzony
	w następujący sposób:
	\begin{enumerate}
		\item
			Duże litery alfabetu zostaną zamienione na odpowiadające im
			małe litery. Z tekstu zostaną usunięte liczby oraz
			znaki przestankowe	
		
		\item
			Tekst zostanie poddany \emph{stemmingowi},
			dzięki któremu zostaną
			wyodrębnione rdzenie poszczególnych słów. 
			Na przykład słowa:
			\textit{fishing}, \textit{fished} oraz \textit{fisher} 
			zostaną zamienione na słowo \textit{fish}.
			Stemming zostanie przeprowadzony za pomocą 
			algorytmu  Portera, dostępnego między innymi
			w pakiecie \texttt{tm} języka R
			
		\item
			Zostaną usunięte słowa nie niosące żadnej informacji
			na przykład słówko \textit{the}, dodatkowo zostaną 
			usunięte słowa występujące zbyt rzadko w streszczeniach --
			odpowiedni parametr zostanie dobrany eksperymentalnie 		
	\end{enumerate}
	
	Następnie dokonamy wektoryzacji tekstów. Niech $D$ oznacza 
	zbiór słów które występują w przynajmniej jednym z przetworzonych
	tekstów, a $T$ zbiór wszystkich tekstów. 
	Wtedy tekstowi $t \in T$ będzie odpowiadał wektor $v_t$
	indeksowany słowami $w \in D$ taki że 
	\[ v_t[w] = \textrm{tf-idf}(w, t) \]
	Gdzie wartość $\textrm{tf-idf}(w, t)$ jest zdefiniowana następująco
	\[ \textrm{tf-idf}(w, t) = \textrm{tf}(w, t) \cdot log{\frac{|T|}{|\{t \in T: w \in t \}| + 1}} \]
	przy czym tf$(w, t)$ oznacza liczbę wystąpień słowa $w$ w tekście $t$.
	
	Przedstawione w następnej sekcji algorytmy operować będą
	na utworzonym w powyższy sposób zbiorze wektorów $\{v_t: t \in T\}$.

\section{Algorytmy}

\subsection{Naiwny klasyfikator bayesowski}
Naiwna metoda Bayesa, jest probabilistyczną metodą opracowaną przez Thomasa
Bayesa w XVIII wieku. Podstawowym założeniem metody jest rozłączność
(niezależność) cech obiektu. Dlatego nazywamy tą metodą naiwnym
klasyfikatorem, ponieważ jest to mocne uproszczenie rzeczywistości
praktycznie niespotykane. Następnie wyliczane jest \textit{a priori}
prawdopodobieństwo poszczególnych grup i atrybutów. W praktyce często
korzysta się z  \textit{maximum likelihood estimation}, rzadziej przy pomocy
estymatora Bayseowskiego.

Algorytm można opisać równaniem:

\[
\max_{c \in G} P(c | X) = \frac{P(X|c)P(c)}{P(X)}
\]
gdzie:

$P(c)$ - prawdopodobieństwo \textit{a priori} przynależenia do danej klasy

$X = (x_1,x_2,...,x_n)$ - Wektor atrybutów 

$P(c) = s_i / s$, $s$ ilość obiektów z zbiorze treningowym, $s_i$ ilość obiektów w danej grupie. 

$P(X|c) = \prod\limits_{i} P(x_i | c)$

$P(x_i | c) = s_{ic} / s_i$, gdzie $s_{ic}$ to ilość obiektów o wartości atrybutu równej $x_i$ w danej klasie $c$.


\subsection{kNN}
K najbliższych sąsiadów (k nearest neighbours), jest leniwym algorytmem
gęstościowym. Jest to leniwy algorytm ponieważ, wszystkie najważniejsze
obliczenia są wykonywane w momencie klasyfikacji. Trening modelu polega na
umieszczenie wszystkich obiektów treningowy w przestrzeni atrybutów. Z kolei
podczas klasyfikacji, badany obiekt rzutujemy do danej przestrzeni i szukamy
jego k-najbliższych sąsiadów. Następnie sąsiedzi głosuje zgodnie ze swoją
przynależnością grupową, w ten sposób określając grupę klasyfikowanego
obiektu. 

\paragraph{} Sam algorytm jest bardzo prosty i zarazem bardzo skuteczny. Jego
specyfikę określają wartość $k$ oraz metryka opisująca odległość miedzy
obiektami. Najłatwiej jego działanie przedstawić na przykładnie pseudo kodu.

\begin{verbatim}
PriorityQueue pq
for t in traning_objects:
    pq.add( dist(Obj,t) )

PriorityMap m
for t in pq.get(k):
    if t in m:
        m[t] = 1
        continue
    m[t] += 1 

return m.getMax().group
\end{verbatim}

Gzie \textit{Obj} to klasyfikowany obiekt a \textit{k} to ilość
rozpatrywanych sąsiadów i \textit{dist} jest funkcją metryki odległości.

\section{Eksperymenty}
\paragraph{}
Celem projektu jest implementacja wymienionych powyżej algorytmów i porównanie ich z już dostępnymi w języku R. Dlatego eksperymenty będą polegały na zbadaniu skuteczności poszczególnych algorytmów dla różnych zestawów uczących i testujących. 


\paragraph{}
Wybrany zbiór danych stanowi streszczenia projektów badawczych. Jest to
ciekawy zbiór danych ponieważ teksty naukowe zawierają zestawy słów
unikalnych dla konkretnych dziedzin problemu. Dlatego spodziewane są bardzo
dobre wyniki klasyfikacji dla każdego z algorytmu. 

Dodatkowo zbiór słów w dokumencie zostanie poddany stemmingowi, czyli
wyekstrahowaniu rdzeni słów co również umożliwi dokładniejszą klasyfikację.
Zbiór danych jest w języku angielskim dlatego w ramach projektu wykorzystamy
stemmer Portera.  


\paragraph{}
Jakość modeli zostanie przedstawiona w postaci wartości procentowej
opisującej ilość poprawnie zaklasyfikowanych obiektów w danej próbie. Dla
potrzeb porównawczych zostanie również przestawiona średnia wartość ze
wszystkich testów dla każdego algorytmu. 

% --------------------------------------------
% BIBLIOGRAFIA
% --------------------------------------------

\begin{thebibliography}{9}

\bibitem{abstracts}
   \emph{NSF Research Award Abstracts 1990-2003 Data Set}.
   \url{http://archive.ics.uci.edu/ml/datasets/NSF+Research+Award+Abstracts+1990-2003}
\bibitem{irbook}
	\emph{An Introduction to Information Retrieval}. 
	C. D. Manning, P. Raghavan, H. Sch\"utze.
	Cambridge University Press.
	Wersja online: \url{http://nlp.stanford.edu/IR-book/html/htmledition/irbook.html}
\bibitem{wikistemming}
	Wikipedia, Stemming, \url{http://en.wikipedia.org/wiki/Stemming}

\end{thebibliography}


\end{document}
